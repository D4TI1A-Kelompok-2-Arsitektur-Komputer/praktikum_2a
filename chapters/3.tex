%%%%%%%%%%%%%%%%%%%%%%%%%%%%%%%%%%%%%%%%%%%%%%%%%%%%%%%%%%%%%%
\section{Harun Ar - Rasyid}
\subsubsection{Pemahanan Teori}
\begin{enumerate}
    \item Apa itu fungsi, inputan fungsi dan kembalian fungsi dengan contoh kode program
    lainnya.
    Fungsi adalah bagian dari program yang dapat digunakan ulang.
    Berikut merupakan contoh fungsi dan cara pemanggilannya
    \lstinputlisting[firstline=124, lastline=127]{src/1174027.py}

    Fungsi dapat membaca parameter, parameter adalah nilai yang disediakan kepada fungsi, dimana nilai ini akan menentukan output yang akan dihasilkan fungsi.
    \lstinputlisting[firstline=129, lastline=132]{src/1174027.py}

    Statemen return digunakan untuk keluar dari fungsi. Kita juga dapat menspesifikasikan nilai kembalian.
    \lstinputlisting[firstline=134, lastline=141]{src/1174027.py}

    \item Apa itu paket dan cara pemanggilan paket atau library dengan contoh kode
    program lainnya.
    Untuk memudahkan dalam pemanggilan fungsi yang di butuhkan, agar dapat dipanggil berulang.
    Cara pemanggilannya
    \lstinputlisting[firstline=143, lastline=144]{src/1174027.py}

    \item Jelaskan Apa itu kelas, apa itu objek, apa itu atribut, apa itu method dan
    contoh kode program lainnya masing-masing.
    kelas merupakan sebuah blueprint yang mepresentasikan objek.
    objek adalah hasil cetakan dadri sebuah kelas.
    method adalah suatu upaya yang digunakan oleh object.
    \lstinputlisting[firstline=146, lastline=168]{src/1174027.py}

    \item Jelaskan cara pemanggikan library kelas dari instansiasi dan pemakaiannya den-
    gan contoh program lainnya.
    Cara Pemanggilanya 
    \begin{itemize}
        \item pertama import terlebih dahulu filenya.
        \item kemudian buat variabel untuk menampung datanya
        \item setelah itu panggil nama classnya dan panggil methodnya
        \item Gunakan perintah print untuk menampilkan hasilnya

    \end{itemize}
    \lstinputlisting[firstline=170, lastline=175]{src/1174027.py}

    \item Jelaskan dengan contoh pemakaian paket dengan perintah from kalkulator im-
    port Penambahan disertai dengan contoh kode lainnya.
    Penggunaan paket from namafile import, itu berfungsi untuk memanggil file dan fungsinya
    \lstinputlisting[firstline=143, lastline=144]{src/1174027.py}

    \item Jelaskan dengan contoh kodenya, pemakaian paket fungsi apabila le library
    ada di dalam folder.
    Pemakaian paket adalah perkumpulan fungsi-fungsi. contoh kodenya adalah sebagai berikut :

    \item Jelaskan dengan contoh kodenya, pemakaian paket kelas apabila le library ada
    di dalam folder.
    \lstinputlisting[firstline=184, lastline=184]{src/1174027.py}

\end{enumerate}
\subsubsection{Ketrampilan Pemrograman}
\begin{enumerate}
    \item Buatlah fungsi dengan inputan variabel NPM, dan melakukan print luaran huruf
    yang dirangkai dari tanda bintang, pagar atau plus dari NPM kita. Tanda
    bintang untuk NPM mod 3=0, tanda pagar untuk NPM mod 3 =1, tanda plus
    untuk NPM mod3=2.
    \lstinputlisting[firstline=184, lastline=234]{src/1174027.py}

    \item Buatlah fungsi dengan inputan variabel berupa NPM. kemudian dengan meng-
    gunakan perulangan mengeluarkan print output sebanyak dua dijit belakang
    NPM.
    \lstinputlisting[firstline=237, lastline=243]{src/1174027.py}

    \item Buatlah fungsi dengan dengan input variabel string bernama NPM dan beri
    luaran output dengan perulangan berupa tiga karakter belakang dari NPM se-
    banyak penjumlahan tiga dijit tersebut.
    \lstinputlisting[firstline=245, lastline=255]{src/1174027.py}

    \item Buatlah fungsi hello word dengan input variabel string bernama NPM dan
    beri luaran output berupa digit ketiga dari belakang dari variabel NPM meng-
    gunakan akses langsung manipulasi string pada baris ketiga dari variabel NPM.
    \lstinputlisting[firstline=257, lastline=263]{src/1174027.py}

    \item buat fungsi program dengan input variabel NPM dan melakukan print nomor npm satu persatu kebawah.
    \lstinputlisting[firstline=265, lastline=269]{src/1174027.py}

    \item Buatlah fungsi dengan inputan variabel NPM, didalamnya melakukan penjum-
    lahan dari seluruh dijit NPM tersebut, wajib menggunakan perulangan dan
    atau kondisi.
    \lstinputlisting[firstline=272, lastline=279]{src/1174027.py}

    \item Buatlah fungsi dengan inputan variabel NPM, didalamnya melakukan melakukan
    perkalian dari seluruh dijit NPM tersebut, wajib menggunakan perulangan dan
    atau kondisi.
    \lstinputlisting[firstline=281, lastline=288]{src/1174027.py}

    \item Buatlah fungsi dengan inputan variabel NPM, Lakukan print NPM anda tapi
    hanya dijit genap saja. wajib menggunakan perulangan dan atau kondisi.
    \lstinputlisting[firstline=290, lastline=296]{src/1174027.py}

    \item Buatlah fungsi dengan inputan variabel NPM, Lakukan print NPM anda tapi
    hanya dijit ganjil saja. wajib menggunakan perulangan dan atau kondisi.
    \lstinputlisting[firstline=298, lastline=304]{src/1174027.py}

    \item Buatlah fungsi dengan inputan variabel NPM, Lakukan print NPM anda tapi
    hanya dijit yang termasuk bilangan prima saja. wajib menggunakan perulangan
    dan atau kondisi.
    \lstinputlisting[firstline=306, lastline=320]{src/1174027.py}

    \item Buatlah satu library yang berisi fungsi-fungsi dari nomor diatas dengan nama
    le 3lib.py dan berikan contoh cara pemanggilannya pada le main.py.
    \lstinputlisting[firstline=7, lastline=7]{src/main.py}

    \item Buatlah satu library class dengan nama le kelas3lib.py yang merupakan mod-
    ikasi dari fungsi-fungsi nomor diatas dan berikan contoh cara pemanggilannya
    pada le main.py.
    \lstinputlisting[firstline=8, lastline=9]{src/main.py}
    
\end{enumerate}
\subsubsection{Ketrampilan Penanganan Error}
Error yang di dapat dari mengerjakan tugas ini adalah type error, cara menaggulaginya dengan cara mengecheck kembali codingannya
kemudian run kembali aplikasinya
berikut contoh Penggunaan fungsi try dan exception
\lstinputlisting[firstline=177, lastline=182]{src/1174027.py}
%%%%%%%%%%%%%%%%%%%%%%%%%%%%%%%%%%%%%%%%%%%%%%%%%%%%%%%%%%%%%%
\section{Evietania}
\subsubsection{Pemahanan Teori}
\begin{enumerate}
    \item Apa itu fungsi, inputan fungsi dan kembalian fungsi dengan contoh kode program
    lainnya.
    Fungsi adalah bagian dari program yang dapat digunakan ulang.
    Berikut merupakan contoh fungsi dan cara pemanggilannya
    \lstinputlisting[firstline=124, lastline=127]{src/1174051_praktek.py}

    Fungsi dapat membaca parameter, parameter adalah nilai yang disediakan kepada fungsi, dimana nilai ini akan menentukan output yang akan dihasilkan fungsi.
    \lstinputlisting[firstline=129, lastline=132]{src/1174051_praktek.py}

    Statemen return digunakan untuk keluar dari fungsi. Kita juga dapat menspesifikasikan nilai kembalian.
    \lstinputlisting[firstline=134, lastline=141]{src/1174051_praktek.py}

    \item Apa itu paket dan cara pemanggilan paket atau library dengan contoh kode
    program lainnya.
    Untuk memudahkan dalam pemanggilan fungsi yang di butuhkan, agar dapat dipanggil berulang.
    Cara pemanggilannya
    \lstinputlisting[firstline=143, lastline=144]{src/1174051_praktek.py}

    \item Jelaskan Apa itu kelas, apa itu objek, apa itu atribut, apa itu method dan
    contoh kode program lainnya masing-masing.
    kelas merupakan sebuah blueprint yang mepresentasikan objek.
    objek adalah hasil cetakan dadri sebuah kelas.
    method adalah suatu upaya yang digunakan oleh object.
    \lstinputlisting[firstline=146, lastline=168]{src/1174051_praktek.py}

    \item Jelaskan cara pemanggikan library kelas dari instansiasi dan pemakaiannya den-
    gan contoh program lainnya.
    Cara Pemanggilanya 
    \begin{itemize}
        \item pertama import terlebih dahulu filenya.
        \item kemudian buat variabel untuk menampung datanya
        \item setelah itu panggil nama classnya dan panggil methodnya
        \item Gunakan perintah print untuk menampilkan hasilnya

    \end{itemize}
    \lstinputlisting[firstline=170, lastline=175]{src/1174051_praktek.py}

    \item Jelaskan dengan contoh pemakaian paket dengan perintah from kalkulator im-
    port Penambahan disertai dengan contoh kode lainnya.
    Penggunaan paket from namafile import, itu berfungsi untuk memanggil file dan fungsinya
    \lstinputlisting[firstline=143, lastline=144]{src/1174051_praktek.py}

    \item Jelaskan dengan contoh kodenya, pemakaian paket fungsi apabila le library
    ada di dalam folder.
    Pemakaian paket adalah perkumpulan fungsi-fungsi. contoh kodenya adalah sebagai berikut :

    \item Jelaskan dengan contoh kodenya, pemakaian paket kelas apabila le library ada
    di dalam folder.
    \lstinputlisting[firstline=184, lastline=184]{src/1174051_praktek.py}

\end{enumerate}
\subsubsection{Ketrampilan Pemrograman}
\begin{enumerate}
    \item Buatlah fungsi dengan inputan variabel NPM, dan melakukan print luaran huruf
    yang dirangkai dari tanda bintang, pagar atau plus dari NPM kita. Tanda
    bintang untuk NPM mod 3=0, tanda pagar untuk NPM mod 3 =1, tanda plus
    untuk NPM mod3=2.
    \lstinputlisting[firstline=184, lastline=234]{src/1174051_praktek.py}

    \item Buatlah fungsi dengan inputan variabel berupa NPM. kemudian dengan meng-
    gunakan perulangan mengeluarkan print output sebanyak dua dijit belakang
    NPM.
    \lstinputlisting[firstline=237, lastline=243]{src/1174051_praktek.py}

    \item Buatlah fungsi dengan dengan input variabel string bernama NPM dan beri
    luaran output dengan perulangan berupa tiga karakter belakang dari NPM se-
    banyak penjumlahan tiga dijit tersebut.
    \lstinputlisting[firstline=245, lastline=255]{src/1174051_praktek.py}

    \item Buatlah fungsi hello word dengan input variabel string bernama NPM dan
    beri luaran output berupa digit ketiga dari belakang dari variabel NPM meng-
    gunakan akses langsung manipulasi string pada baris ketiga dari variabel NPM.
    \lstinputlisting[firstline=257, lastline=263]{src/1174051_praktek.py}

    \item buat fungsi program dengan input variabel NPM dan melakukan print nomor npm satu persatu kebawah.
    \lstinputlisting[firstline=265, lastline=269]{src/1174051_praktek.py}

    \item Buatlah fungsi dengan inputan variabel NPM, didalamnya melakukan penjum-
    lahan dari seluruh dijit NPM tersebut, wajib menggunakan perulangan dan
    atau kondisi.
    \lstinputlisting[firstline=272, lastline=279]{src/1174051_praktek.py}

    \item Buatlah fungsi dengan inputan variabel NPM, didalamnya melakukan melakukan
    perkalian dari seluruh dijit NPM tersebut, wajib menggunakan perulangan dan
    atau kondisi.
    \lstinputlisting[firstline=281, lastline=288]{src/1174051_praktek.py}

    \item Buatlah fungsi dengan inputan variabel NPM, Lakukan print NPM anda tapi
    hanya dijit genap saja. wajib menggunakan perulangan dan atau kondisi.
    \lstinputlisting[firstline=290, lastline=296]{src/1174051_praktek.py}

    \item Buatlah fungsi dengan inputan variabel NPM, Lakukan print NPM anda tapi
    hanya dijit ganjil saja. wajib menggunakan perulangan dan atau kondisi.
    \lstinputlisting[firstline=298, lastline=304]{src/1174051_praktek.py}

    \item Buatlah fungsi dengan inputan variabel NPM, Lakukan print NPM anda tapi
    hanya dijit yang termasuk bilangan prima saja. wajib menggunakan perulangan
    dan atau kondisi.
    \lstinputlisting[firstline=306, lastline=320]{src/1174051_praktek.py}

    \item Buatlah satu library yang berisi fungsi-fungsi dari nomor diatas dengan nama
    le epi.py dan berikan contoh cara pemanggilannya pada le main.py.
    \lstinputlisting[firstline=7, lastline=7]{src/mainn.py}

    \item Buatlah satu library class dengan nama le kelas3lib.py yang merupakan mod-
    ikasi dari fungsi-fungsi nomor diatas dan berikan contoh cara pemanggilannya
    pada le mainn.py.
    \lstinputlisting[firstline=8, lastline=9]{src/mainn.py}
    
\end{enumerate}
\subsubsection{Ketrampilan Penanganan Error}
Error yang di dapat dari mengerjakan tugas ini adalah type error, cara menaggulaginya dengan cara mengecheck kembali codingannya
kemudian run kembali aplikasinya
berikut contoh Penggunaan fungsi try dan exception
\lstinputlisting[firstline=177, lastline=182]{src/1174051_praktek.py}

%%%%%%%%%%%%%%%%%%%%%%%%%%%%%%%%%%%%%%%%%%%%%%%%%%%%%%%%%%%%%%
\section{Kadek Diva Krishna Murti}
\subsection{Pemahaman Teori}
\begin{enumerate}
	
	%No. 1
	\item Pengertian fungsi, inputan fungsi, dan kembalian fungsi serta contoh kode programnya.
	
	\begin{itemize}
		
		\item Fungsi adalah blok program untuk melakukan tugas-tugas tertentu yang dilakukan berulang dan dapat digunakan berulang kali dari tempat lain di dalam program.
		\lstinputlisting[caption = namaFungsi merupakan nama dari fungsi yang dibuat., firstline=10, lastline=10]{src/1174006/Chapter3/1174006.py}
		
		\item Inputan fungsi adalah inputan yang berasal dari luar fungsi yang akan di proses di dalam fungsi itu sendiri.
		\lstinputlisting[caption = inputanFungsi merupakan nama dari inputan fungsi yang diterima dari luar fungsi namaFungsi., firstline=10, lastline=10]{src/1174006/Chapter3/1174006.py}
		
		\item Kembalian fungsi adalah untuk mengembalikan suatu nilai ekspresi dari proses yang dilakukan fungsi.
		\lstinputlisting[caption = return inputanFungsi merupakan kembalian fungsi dari fungsi namaFungsi., firstline=11, lastline=11]{src/1174006/Chapter3/1174006.py}
		
	\end{itemize}
	
	Penggunaan fungsi di Python
	\lstinputlisting[caption = Contoh penggunaan fungsi di Python., firstline=9, lastline=14]{src/1174006/Chapter3/1174006.py}
	
	%No. 2
	\item Pengertian paket dan cara pemanggilannya serta contoh kode programnya.
	
	Paket atau library adalah file yang berisi kode program python yang bisa digunakan berulang dimana paket itu dipanggil.
	
	Cara pemanggilan paket atau library yaitu dengan meng-import paket atau library yang akan digunakan. Lalu panggil dengan cara mendefinisikan namapaket.namafungsinya.
	
	Berikut ini merupakan contoh penggunaan paket atau library.
	\lstinputlisting[caption = Contoh penggunaan paket atau library., firstline=16, lastline=18]{src/1174006/Chapter3/1174006.py}
	
	%No. 3
	\item Pengertian kelas, objek, atribut, method, dan contoh kode programnya.
	
	\begin{itemize}
		\item Kelas
		Kelas adalah cetak biru atau prototipe dari objek dimana kita mendefinisikan atribut dari suatu objek.
		Contoh penggunaan kelas di python.
		\lstinputlisting[caption = Contoh penggunaan kelas di python., firstline=20, lastline=40]{src/1174006/Chapter3/1174006.py}
		
		\item Objek
		Objek adalah instansi atau perwujudan dari sebuah kelas.
		\lstinputlisting[caption = Contoh penggunaan objek di python., firstline=34, lastline=35]{src/1174006/Chapter3/1174006.py}
		
		\item Atribut
		Atribut adalah variabel yang menyimpan data yang berhubungan dengan kelas dan objeknya.
		\lstinputlisting[caption = Contoh penggunaan atribut di python., firstline=22, lastline=22]{src/1174006/Chapter3/1174006.py}
		
		\item Method
		Metode adalah fungsi yang didefinisikan di dalam suatu kelas.
		\lstinputlisting[caption = Contoh penggunaan method di python., firstline=29, lastline=32]{src/1174006/Chapter3/1174006.py}
		
	\end{itemize}
	
	%No. 4
	\item Cara pemanggilan library kelas, dan contoh kode programnya.
	
	Berikut ini adalah cara pemanggilan library kelas dari instansi dan pemakaiannya. Library kelasnya adalah Mahasiswa dari file Mahasiswa.py. Lalu dipanggil dengan import. Kemudian instansi dengan mhs1 dan mhs1, dengan 2 parameter.
	\lstinputlisting[caption= Contoh pemanggilan library kelas dari instansi dan pemakaiannya. .,firstline=42, lastline=51]{src/1174006/Chapter3/1174006.py}
	
	%No. 5
	\item Penjelasan pemakaian paket disertai dengan contoh kode programnya.
	
	Berikut ini adalah contoh pemakaian paket dengan perintah from kalkulator import Penambahan. Setelah mengimport paketnya, lalu panggil fungsi penambahannya.
	\lstinputlisting[caption= Contoh pemakaian paket dengan perintah from kalkulator import Penambahan.,firstline=53, lastline=57]{src/1174006/Chapter3/1174006.py}
	
	%No. 6
	\item Contoh kode pemakaian paket fungsi apabila file library ada di dalam folder. Berikut ini adalah pemakaian paket fungsi apabila file library ada di dalam folder.
	\lstinputlisting[caption= Contoh kode pemakaian paket fungsi dimana file library ada di dalam folder., firstline=59, lastline=73]{src/1174006/Chapter3/1174006.py}
	
	%No. 7
	\item Contoh kode pemakaian paket kelas apabila file library ada di dalam folder. Berikut ini adalah pemakaian paket kelas apabila file library ada di dalam folder.
	\lstinputlisting[caption= Contoh kode pemakaian paket kelas dimana file library ada di dalam folder., firstline=75, lastline=84]{src/1174006/Chapter3/1174006.py}
\end{enumerate}

\subsection{Ketrampilan Pemrograman}
\begin{enumerate}
	\item Jawaban soal No. 1
	\lstinputlisting[caption = Jawaban soal No. 1 Ketrampilan Pemrograman., firstline=87, lastline=121]{src/1174006/Chapter3/1174006.py}
	
	\item Jawaban soal No. 2
	\lstinputlisting[caption = Jawaban soal No. 2 Ketrampilan Pemrograman., firstline=123, lastline=131]{src/1174006/Chapter3/1174006.py}
	
	\item Jawaban soal No. 3
	\lstinputlisting[caption = Jawaban soal No. 3 Ketrampilan Pemrograman., firstline=133, lastline=142]{src/1174006/Chapter3/1174006.py}
	
	\item Jawaban soal No. 4
	\lstinputlisting[caption = Jawaban soal No. 4 Ketrampilan Pemrograman., firstline=144, lastline=149]{src/1174006/Chapter3/1174006.py}
	
	\item Jawaban soal No. 5
	\lstinputlisting[caption = Jawaban soal No. 5 Ketrampilan Pemrograman., firstline=151, lastline=157]{src/1174006/Chapter3/1174006.py}
	
	\item Jawaban soal No. 6
	\lstinputlisting[caption = Jawaban soal No. 6 Ketrampilan Pemrograman., firstline=159, lastline=168]{src/1174006/Chapter3/1174006.py}
	
	\item Jawaban soal No. 7
	\lstinputlisting[caption = Jawaban soal No. 7 Ketrampilan Pemrograman., firstline=170, lastline=179]{src/1174006/Chapter3/1174006.py}
	
	\item Jawaban soal No. 8
	\lstinputlisting[caption = Jawaban soal No. 8 Ketrampilan Pemrograman., firstline=181, lastline=189]{src/1174006/Chapter3/1174006.py}
	
	\item Jawaban soal No. 9
	\lstinputlisting[caption = Jawaban soal No. 9 Ketrampilan Pemrograman., firstline=191, lastline=198]{src/1174006/Chapter3/1174006.py}
	
	\item Jawaban soal No. 10
	\lstinputlisting[caption = Jawaban soal No. 10 Ketrampilan Pemrograman., firstline=200, lastline=217]{src/1174006/Chapter3/1174006.py}
	
	\item Jawaban soal No. 11
	\lstinputlisting[caption = Jawaban soal No. 11 Ketrampilan Pemrograman., firstline=30, lastline=43]{src/1174006/Chapter3/main.py}
	
	\item Jawaban soal No. 12
	\lstinputlisting[caption = Jawaban soal No. 12 Ketrampilan Pemrograman., firstline=45, lastline=60]{src/1174006/Chapter3/main.py}
	
\end{enumerate}

\subsection{Ketrampilan Penanganan Error}
\begin{enumerate}
	\item Peringatan error yang ditemukan dan penjelasannya serta buat sebuah fungsi try except untuk menanggulangi error.
	
	Peringatan error di praktek ketiga ini, yaitu:
	\begin{itemize}
		\item Syntax Errors
		Syntax Errors adalah suatu keadaan saat kode python mengalami kesalahan penulisan. Solusinya adalah memperbaiki penulisan kode yang salah.
		
		\item Zero Division Error
		ZeroDivisonError adalah exceptions yang terjadi saat eksekusi program menghasilkan perhitungan matematika pembagian dengan angka nol (0). Solusinya adalah tidak membagi suatu yang hasilnya nol.
		
		\item Name Error
		NameError adalah exception yang terjadi saat kode melakukan eksekusi terhadap local name atau global name yang tidak terdefinisi. Solusinya adalah memastikan variabel atau function yang dipanggil ada atau tidak salah ketik.
		
		\item Type Error
		TypeError adalah exception yang terjadi saat dilakukan eksekusi terhadap suatu operasi atau fungsi dengan type object yang tidak sesuai. Solusinya adalah mengkoversi varibelnya sesuai dengan tipe data yang akan digunakan.
	\end{itemize}
	
	Contoh fungsi yang menggunakan try except
	\lstinputlisting[caption= Contoh fungsi yang menggunakan try except .,firstline=224, lastline=231]{src/1174006/Chapter3/1174006.py}
\end{enumerate}
